\documentclass{article}
\usepackage[utf8]{inputenc}
\usepackage[russian]{babel}
\usepackage{cmap}
\usepackage{amsfonts,amsmath}
\usepackage{amssymb}
\usepackage{geometry}
\geometry{verbose,a4paper,tmargin=1.0cm,bmargin=1.6cm,lmargin=0.6cm,rmargin=0.6cm}
\pdfcompresslevel=9
\usepackage{color}
\usepackage{xcolor}

\begin{document}
  \Huge\textbf{А.Н. Барменков, глава III}\normalsize\\\\

  \huge\textbf{Длина дуги плоской кривой}\normalsize\\\\
  
  Пусть $\varphi(t)$ и $\psi(t)$ непрерывны на $[\alpha;\beta]$. Рассмотрим плоскость $Oxy$ с ДПСК.\\
  
  \textbf{Определение:} Множество точек $M(x;y)$ плоскости, координаты которых удовлетворяют соотношению $x=\varphi(t);\;\;y=\psi(t);\;\;\alpha\leq t\leq\beta$ называют \textit{простой плоской кривой} $\Gamma$, если различные значения параметра $t\in[\alpha;\beta]$ соответствуют различным точкам этого множества \textit{(простая кривая --- кривая без самопересечений)}.\\
  
  Каждую точку $M(x;y)$, координаты которой соответствуют значениям параметра $t\in(\alpha;\beta)$ считают точкой кривой, а точки, отвечающие граничным значениям $\alpha$ и $\beta$ --- граничными точками кривой ($M[\varphi(\alpha);\psi(\alpha)]$ --- начальная точка кривой, $M[\varphi(\beta);\psi(\beta)]$ --- конечная точка кривой)\\
  
  \textbf{Определение:} простой замкнутой кривой называют простую кривую, у которой начальная и конечная точки совпадают.\\
  
  Пусть $\tau$ --- произвольное разбиение $[\alpha;\beta]$:
  \begin{equation}
    \tau=\{t_i\}^n_{i=0}\;\;(\alpha=t_0<t_1<...<t_n=\beta)
  \end{equation}
  
  Введем $M_i$:
  \begin{equation}
    M_i=M[\varphi(t_i);\psi(t_i)]
  \end{equation}
  
  Соединив точки отрезками, получим \textit{ломаную, вписанную в кривую $\Gamma$.} Длина звена ломаной:
  \begin{equation}
    l_i=|M_{i-1}M_i|=\sqrt{[\varphi(t_i)-\varphi(t_{i-1})]^2+[\psi(t_i)-\psi(t_{i-1})]^2}
  \end{equation}
  
  Тогда длина всей ломаной:
  \begin{equation}
    l(\tau)=\sum_{i=1}^n\sqrt{[\varphi(t_i)-\varphi(t_{i-1})]^2+[\psi(t_i)-\psi(t_{i-1})]^2}
  \end{equation}
  
  \textbf{Определение:} если множество $l(\tau)$ вписанных в кривую $\Gamma$ ломанных, соответствующих всевозможным разбиениям $\tau$ отрезка $[\alpha,\beta]$ ограничено, то кривая $\Gamma$ называется \textit{спрямляемой}, а точная верхняя грань множества $l(\tau)$ называется длиной кривой $\Gamma$:
  \begin{equation}
    l=\sup(l(\tau))
  \end{equation}
  
  \textbf{Замечание:} из определения длины дуги $l$ следует, что $l>0$.\\
  
  \textbf{Замечание:} существуют и неспрямляемые кривые.\\
  
  \large\textbf{Некоторые свойства спрямляемых кривых}\normalsize\\
  
  1$^\circ$ Если кривая $\Gamma$ --- спрямляема, и $l$ --- ее длина, то эта длина дуги не зависит от параметризации этой кривой.\\
  
  2$^\circ$ Если спрямляемая кривая $\Gamma$ разбита при помощи конечного числа точек $M_0, M_1, ... M_n$ на конечное число кривых $\Gamma_i$, то каждая из этих кривых спрямляема, и сумма длин всех кривых $\Gamma_i$ равна длине $l$ кривой $\Gamma$.\\
  
  3$^\circ$ Пусть кривая $\Gamma$ задана параметрическими уравнениями $(1)$. Обозначим $l(t)$ --- длину участка $\Gamma_t$ кривой $\Gamma$, точки которой определяются всеми значениями параметра из сегмента $[\alpha;t]$. Функция $l(t)$ возрастающая и непрерывная. Эту функцию называют \textit{переменной дугой} на кривой $\Gamma$\\
  
  4$^\circ$ Переменная дуга $l$ может быть выбрана в качестве параметра, называемого \textit{натуральным параметром}.\\
  
  \textbf{Утверждение:} если функции $x=\varphi(t), y=\psi(t)$ имеют на $[\alpha,\beta]$ непрерывные производные, то кривая $\Gamma$, определяемая вышеописанными соотношениями \textit{спрямляема.}\\
  
  \textbf{Доказательство:} рассмотрим произвольное разбиение $\tau$ отрезка $[\alpha;\beta]$:
  \begin{equation}
    \tau=\{t_i\}_{i=0}^n, \;\;\; \alpha=t_0<t_1<...<t_n=\beta
  \end{equation}
  
  Длина ломаной $M_0M_1...M_n$, вписанной в $\gamma$:
  \begin{equation}
    l(\tau)=\sum_{i=1}^n \sqrt{(\varphi(t_i)-\varphi(t_{i-1}))^2+(\psi(t_i)-\psi(t_{i-1}))^2}
  \end{equation}
  
  По теореме Лагранжа о конечных приращениях:
  \begin{equation}
    \forall[t_{i-1};t_i] \exists \xi_i, \xi_i*\in[t_{i-1};t_i]:\;\;\;\; \varphi(t_i)-\varphi(t_{i-1})=\varphi'(\xi_i)\Delta t_i,\;\;\;\; \psi(t_i)-\psi(t_{i-1})=\psi'(\xi_i*)\Delta t_i
  \end{equation}
  
  Поскольку $\varphi'(t),\psi'(t)$ непрерывны на $[\alpha;\beta]$, то они ограничены на $[\alpha;\beta]$, т.е. $\exists M:|\varphi'(t)|\leq M, |\psi'(t)|\leq M$,
  
  \begin{equation}
    l(\tau)=\sum_{i=1}^n\sqrt{(\varphi'(\xi_i))^2+(\psi'(\xi_i*))^2}\Delta t_i\leq\sum_{i=1}^n\sqrt(M^2+M^2)\Delta t_i=M\sqrt{2}\sum_{i=1}^n\Delta t_i=M\sqrt{2}(\beta-\alpha)
  \end{equation}
  
  \textbf{Утверждение доказано.}\\
  
  \huge\textbf{Пространственная кривая}\normalsize\\
  
  \textbf{Определение:} простой пространственной кривой $\Gamma$ называется геометрическое место точек $M(x;y;z)$ пространства $O_{xyz}$, координаты которых $(x;y;z)$ удовлетворяют соотношению $x=\varphi(t); y=\psi(t); z=\chi(t)$, $t\in[\alpha;\beta]$, где $\varphi(t), \psi(t), \chi(t)$ --- непрерывные функции, и эта кривая без \textit{самопересечения}. Если $M[\varphi(\alpha),\psi(\alpha),\chi(\alpha)]=M[\varphi(\beta),\psi(\beta),\chi(\beta)]$, то кривая называется \textit{замкнутой простой кривой}.\\\\
  
  Рассмотрим произвольное разбиение $[\alpha,\beta]$ 
  \begin{equation}
    \tau=\{\alpha=t_0<t_1<...<t_n=\beta\}
  \end{equation}
  
  \begin{equation}
    M_i=M[\varphi(t_i),\psi(t_i),\chi(t_i)]
  \end{equation}

  --- узловые точки $\Gamma$, соответствующие данному разбиению.\\
  
  Тогда   \textbf{Длина ломаной, соответствующей данному разбиению:}
  
  \begin{equation}
    l(\tau)=\sum_{i=1}^n\sqrt{[\varphi(t_i)-\varphi(t_{i-1})]^2+[\psi(t_i)-\psi(t_{i-1})]^2+[\chi(t_i)-\chi(t_{i-1})]^2}
  \end{equation}
  
  \textbf{Определение:} простая кривая $\Gamma$ называется \textit{спрямляемой}, если длины $l(\tau)$ всех ломаных, вписанных в кривую $\Gamma$, соответствующих всевозможных разбиений $\tau=\{t_i\}^n_{i=0}$, вписанных в $\Gamma$ в сумме --- ограниченное множество.\\
  
  \textbf{Длина кривой $l$}:
  \begin{equation}
    l=\sup_\tau{l(\tau)}
  \end{equation}
  
  Аналогично плоскому случаю, \textit{достаточным условием спрямления $\Gamma$} является \textit{непрерывная дифференцируемость} функций $\varphi(t), \psi(t), \chi(t)$.\\
  
  \textbf{Замечание 1:} Имеет место неравенство:
  \begin{equation}
    \sqrt{(\int_a^b f_1(t)dt)^2+(\int_a^b f_2(t)dt)^2+(\int_a^b f_3(t)dt)^2}\leq\int_a^b\sqrt{(f_1(t))^2+(f_2(t))^2+(f_3(t))^2}dt
  \end{equation}
  
  \textbf{Замечание 2:} Неравенство треугольника:\\
  
  Пусть задан треугольник с вершинами
  \begin{equation}
    O_{xyz}:\;\; O(0,0,0),\;\; A(a_1,a_2,a_3),\;\; B(b_1,b_2,b_3)
  \end{equation}
  
  Тогда:
  \begin{equation}
    |\sqrt{\sum_{i=1}^3a_i^2}-\sqrt{\sum_{i=1}^3b_i^2}|\leq\sqrt{\sum_{i=1}^3(a_i-b_i)^2}
  \end{equation}
  
  \textit{(Разность длин сторон всегда меньше либо равна третьей стороне)}\\

  \textbf{Теорема:} пусть функции $x=\varphi(t)$, $y=\psi(t)$, $z=\chi(t)$ непрерывно дифференцируемы (т.е. имеют непрерывную производную на $[a,b]$), тогда кривая $\Gamma$ ($x=\varphi(t)$, $y=\psi(t)$, $z=\chi(t)$, $t\in[a;b]$) \textit{спрямляема} и ее длина выражается в виде:
  \begin{equation}
    l=\int_a^b\sqrt{(\varphi'(t))^2+(\psi'(t))^2+(\chi'(t))^2}dt
  \end{equation}
  
  \textbf{Доказательство:}\\
  
  Докажем, что $\forall\tau=\{t_i\}_{i=0}^n$ --- (разбиения $[a;b]$) ---  соответствующая ему длина ломаной $l(\tau)$ ограничена числом $A$, где $A=\int_a^b\sqrt{(\varphi'(t))^2+(\psi'(t))^2+(\chi'(t))^2}dt$, т.е. докажем, что $l(\tau)\leq A$. 
  \begin{equation}
    l=\int_a^b\sqrt{(\varphi'(t))^2+(\psi'(t))^2+(\chi'(t))^2}dt=\sum_{i=1}^n\sqrt{(\int_{t_{i-1}}^{t_i}\varphi'(t)dt)^2+(\int_{t_{i-1}}^{t_i}\psi'(t)dt)^2+(\int_{t_{i-1}}^{t_i}\chi'(t)dt)^2}
  \end{equation}  
  
  По \textbf{замечанию 1:}
  \begin{equation}
    \sum_{i=1}^n\sqrt{(\int_{t_{i-1}}^{t_i}\varphi'(t)dt)^2+(\int_{t_{i-1}}^{t_i}\psi'(t)dt)^2+(\int_{t_{i-1}}^{t_i}\chi'(t)dt)^2}\leq\sum_{i=1}^i\int_{t_{i-1}}^{t_i}\sqrt{(\varphi'(t))^2+(\psi'(t))^2+(\chi'(t))^2}dt
  \end{equation}
  
  \begin{equation}
    =\int_{t_{i-1}}^{t_i}\sqrt{(\varphi'(t))^2+(\psi'(t))^2+(\chi'(t))^2}dt=A
  \end{equation}
  
  Докажем, что на самом деле $A=\sup_{\tau}l(\tau)$. Поскольку $\varphi(t)$, $\psi(t)$, $\chi(t)$ --- непрерывно дифференцируемы, то $\varphi'(t)$, $\psi'(t)$, $\chi'(t)$ непрерывны на $[a;b]$. По теореме Кантора они и равномерно непрерывны на $[a;b]$.
  \begin{equation}
    \forall\varepsilon>0 \exists\delta=\delta(\varepsilon)>0: \forall(t',t''\in[a;b],|t'-t''|<\delta)\Rightarrow
  \end{equation}
  
  \begin{equation}
    |\varphi'(t)-\varphi'(t'')|<\frac{\varepsilon}{\sqrt{3}(b-a)}|\;\;\psi'(t')-\psi'(t'')|<\frac{\varepsilon}{\sqrt{3}(b-a)}\;\;\chi'(t')-\chi'(t'')<\frac{\varepsilon}{\sqrt{3}(b-a)}|
  \end{equation}
  
  Возьмем любое разбиение $\tau=\{t_i\}_{i=0}^n:\;\lambda(\tau)<\delta$ ($\delta$ из предыдущего соотношения). Очевидно: $A-l(\tau)\geq 0$. Оценим $A-l(\tau)$ сверху.
  \begin{equation}
    A-l(\tau)=\int_{t_{i-1}}^{t_i}\sqrt{(\varphi'(t))^2+(\psi'(t))^2+(\chi'(t))^2}dt=A-\sum_{i=1}^n\sqrt{[\varphi(t_i)-\varphi(t_{i-1})]^2+[\psi(t_i)-\psi(t_{i-1})]^2+[\chi(t_i)-\chi(t_{i-1})]^2}
  \end{equation}
  
  По аддитивности разбиваем интеграл на части:
  \begin{equation}
    =\sum_{i=1}^n\int_{t_{i-1}}^{t_i}(\sqrt{(\varphi'(t))^2+(\psi'(t))^2+(\chi'(t))^2}-\sqrt{(\frac{\varphi(t_i)-\varphi(t_{i-1})}{t_i-t_{i-1}})^2+(\frac{\psi(t_i)-\psi(t_{i-1})}{t_i-t_{i-1}})^2+(\frac{\chi(t_i)-\chi(t_{i-1})}{t_i-t_{i-1}})^2})dt
  \end{equation}
  
  Применяем \textbf{замечание 2}:
  \begin{equation}
    \sum_{i=1}^n\int\sqrt{(\varphi'(t)-\frac{\varphi(t_i)-\varphi(t_{i-1})}{t_i-t_{i-1}})^2+(\psi'(t)-\frac{\psi(t_i)-\psi(t_{i-1})}{t_i-t_{i-1}})^2+(\chi'(t)-\frac{\chi(t_i)-\chi(t_{i-1})}{t_i-t_{i-1}})^2}dt
  \end{equation}
  
  По теореме Лагранжа о конечных приращениях:
  \begin{equation}
    \varphi'(\xi)=\frac{\varphi(t_i)-\varphi(t_{i-1})}{t_i-t_{i-1}}
  \end{equation}
  \begin{equation}
    \psi'(\gamma_i)=\frac{\psi(t_i)-\psi(t_{i-1})}{t_i-t_{i-1}}
  \end{equation}
  \begin{equation}
    \chi'(\omega_i)=\frac{\chi(t_i)-\chi(t_{i-1})}{t_i-t_{i-1}}
  \end{equation}
  \begin{equation}
    =\sum_{i=1}^n\int_{t_{i-1}}^{t_i}\sqrt((\varphi'(t)-\varphi'(\xi_i))^2+(\psi'(t)-\psi'(\gamma_i))^2+(\chi'(t)-\chi'(\omega_i))^2)dt
  \end{equation}
  
  Поскольку $\lambda(\tau)<\delta$, применима оценка (по предыдущ. соотн.).
  \begin{equation}
    \leq\sum_{i=1}^n\int_{t_{i-1}}^{t_i}\sqrt{(\frac{\varepsilon}{\sqrt{3}(b-a)})^2+(\frac{\varepsilon}{\sqrt{3}(b-a)})^2+(\frac{\varepsilon}{\sqrt{3}(b-a)})^2}dt=\frac{\varepsilon}{b-a}\sum_{i=1}^n\int_{t_{i-1}}^{t_i}dt=\varepsilon
  \end{equation}
  
  т.е.
  \begin{equation}
    \forall\varepsilon>0\exists\tau=\tau(\varepsilon):A-l(\tau)\leq\varepsilon
  \end{equation}
  
  Это означает, что $A=\sup_\tau l(\tau)$, т.е. длина кривой $\Gamma$ вычисляется по формуле:
  \begin{equation}
    l=\int_a^b\sqrt{(\varphi'(t))^2+(\psi'(t))^2+(\chi'(t))^2}dt
  \end{equation}
  
  \textbf{Теорема доказана.}\\
  
  \huge\textbf{Площадь плоской фигуры}\normalsize\\
  
  Напомним, что \textit{многоугольником} называется часть плоскости, ограниченная простой замкнутой ломаной линией. Понятие площади многоугольника рассмотрено в курсе элементарной математики.\\
  
  \textbf{Определения:} \\
  
  1) Плоской фигурой $Q$ назовем часть плоскости, ограниченную простой замкнутой кривой $l$. Кривую $l$ называют границей фигуры $Q$ (иногда пишут: $L=\partial Q$). \\
  
  2) Многоугольник \textit{вписан} в фигуру $Q$, если каждая точка этого многоугольника принадлежит фигуре $Q$ или ее границе.\\
  
  3) Если все точки плоской фигуры и ее границы принадлежат некоторому многоугольнику, то говорят, что указанный многоугольник \textit{описан} вокруг фигуры $Q$.\\
  
  Ясно, что площадь любого вписанного в фигуру $Q$ многоугольника \textit{не больше} площади любого описанного вокруг фигуры $Q$ многоугольника.\\
  
  Пусть $\{s_i\}$ --- числовое множество площадей вписанных в плоскую фигуру $Q$ многоугольников, а $\{S_d\}$ --- числовое множество площадей, описанных вокруг фигуры $Q$ многоугольников. Очевидно, что множество $\{s_i\}$ ограничено сверху (площадью любого описанного вокруг фигуры $Q$ многоугольника), а множество $\{S_d\}$ ограничено снизу, площадью вписанного многоугольника, или числом $0$. Поскольку эти множества ограничены, у них существуют точные грани. Обозначим $p=\sup \{s_i\}$ --- точная верхняя грань площадей $s_i$ вписанных в $Q$ многоугольников, f $P=inf\{S_d\}$ --- точная нижняя грань площадей $S_d$ многоугольников, описанных вокруг $Q$. Число $p$ --- \textit{нижняя площадь} фигуры $Q$, а $P$ --- верхняя площадь фигуры $Q$.\\
  
  По определению очевидно, что $p\leq P$ для любой фигуры $Q$.\\
  
  \textbf{Определение:} плоская фигура $Q$ называется \textit{квадрируемой}, если верхняя площадь $P$ совпадает с нижней площадью $p$, при этом общее число $\underline P=P=p$ называется площадью фигуры $Q$ (Такое определение ввел Жордан [1838-1922], французский математик).\\
  
  Площадь, как и всякая мера Жордана ($\underline P=\mu$) обладает рядом свойств:\\
  
  1$^\circ$ $\mu(Q)\geq 0 \forall Q$\\
  
  2$^\circ$ $\mu(\square)=1$, площадь квадрата с единичной стороной равна $1$.\\
  
  3$^\circ$ $\mu(Q)$ --- аддитивная функция, т.е. $Q=Q_1\bigcup Q_2,\; Q_1\bigcap Q_2 = \varnothing$\\
  
  4$^\circ$ $\mu(Q)$ --- инвариантно относительно движения $Q$.\\
  
  5$^\circ$ $\mu(Q)$ --- монотонная функция\\

  Эти свойства характерны не только для площадей, но и для объемов. Построение меры Жордана позволяет вычислять значение площади с помощью интегралов Римана.\\
  
  \large\textbf{Критерий квадрируемости фигуры}\normalsize\\
  
  \textbf{Теорема:} для того, чтобы плоская фигура $Q$ была квадрируемой, необходимо и достаточно, чтобы для любого $\varepsilon>0$ можно было указать такой описанный вокруг фигуры $Q$ многоугольник и такой вписанный в фигуру $Q$ многоугольник, что разность их площадей $S_d-s_i<\varepsilon$:
  \begin{equation}
    (Q \text{--- квадр.}) \Leftrightarrow S_d-s_i<\varepsilon\; \forall\varepsilon>0
  \end{equation}
  
  \textbf{Доказательство:}\\
  
  1) Необходимость. Пусть $Q$ --- квадрируемая фигура, т.е. $\underline P=P=p$, тогда $\forall\varepsilon>0$ существует вписанный в фигуру $Q$ многоугольник площади $S_i$ такой, что для $p=\underline P$ можно написать: $\underline P-s_i<\frac{\varepsilon}{2}$ (из определения супремума), и для этого же $\varepsilon$ можно указать такой описанный многоугольник площадь $S_d$ которого отличается от $P=\underline P$ меньше, чем на $\frac{\varepsilon}{2}$, (т.е. $S_d-\underline P<\frac{\varepsilon}{2}$, по определению инфинума). Таким образом $S_d-s_i<\varepsilon$. \textbf{Необходимость доказана.}\\
  
  2) Достаточность. Пусть выбрано произвольное $\varepsilon>0$ и $S_d,\;s_i$ --- площади описанного и вписанного в фигуру $Q$ многоугольников такие, что $S_d-s_i<\varepsilon$. Так как $s_i\leq p \leq \underline P \leq P\leq S_d$, $0\leq S_d-s_i<\varepsilon\;\forall\varepsilon>0$. Отсюда следует, что $P-p\leq S_d-s_i<\varepsilon $, т.е. $\forall\varepsilon>0\Rightarrow p=P$ и $Q$ квадрируема. \textbf{Достаточность доказана. Теорема доказана.}\\\\
  
  \textbf{Определение:} говорят, что граница фигуры $Q$ имеет площадь равную $0$, если для любого $\varepsilon>0$ можно указать такой описанный вокруг фигуры $Q$ многоугольник и такой вписанный в фигуру $Q$ многоугольник, что разность их площадей меньше $\varepsilon$.\\
  
  Это определение позволяет переформулирвать критерий квадрируемости.\\
  
  \textbf{Теорема:} для того, чтобы плоская фигура $Q$ была квадрируемой, необходимо и достаточно, чтобы ее граница имела площадь равную $0$. На самом деле имеет место \textit{достаточный признак квадрируемости плоской фигуры.}, а именно:\\
  
  \textbf{Утверждение:} если граница $l$ плоской фигуры $Q$ является спрямляемой кривой, то фигура $Q$ квадрируема \textit{(без доказательства)}.\\
  
  \large\textbf{Площадь криволинейной трапеции.}\normalsize\\
  
  \textbf{Определение:} \textit{криволинейной трапецией} называется фигура, ограниченная графиком заданной на отрезке $[a;b]$ неотрицательной и непрерывной функции $f(x)$, ординатами, проведенными в точках $a$ и $b$, а также отрезком $[a;b]$ оси $Ox$.\\
  
  \textbf{Предложение:} криволинейная трапеция --- квадрируемая фигура, площадь $\underline P$ которой вычисляется по формуле:
  \begin{equation}
    \underline P = \int_a^b f(x)dx
  \end{equation}
  
  \textbf{Доказательство:} так как $f(x)$ по условию непрерывна на отрезке $[a;b]$, то $f(x)$ и интегрируема на этом отрезке. Тогда по критерию интегрируемости в допредельной форме:
  \begin{equation}
    \forall\varepsilon>0\;\exists\tau=\{x_k\}_{k=0}^\infty\; \text{--- разбиение}\; a=x_1<x_2<...<x_n=b:\; S-s<\varepsilon,
  \end{equation}
  
  Где $S=S(\tau)$ --- врехняя сумма Дарбу, а $s(\tau)$ --- нижняя сумма Дарбу. Поэтому как раз $S_d=S$ --- площадь описанного многоугольника, а $S_i=s$ --- площадь вписанного многоугольника, т.е. $S_d-s_i<\varepsilon$, криволинейная трапеция квадрируема (по критерию квадрируемости).\\
  
  При $\lambda(\tau)\rightarrow 0$ (характеристика разбиения стремится к $0$) получаем:
  \begin{equation}
    s\rightarrow S\rightarrow \int_a^b f(x)dx; \;\; \underline P=\int_a^b f(x)dx
  \end{equation}
  
  В этом и состоит геометрический смысл определенного интеграла. \textbf{Предложение доказано.}\\
  
  \textbf{Замечание 1}: пусть криволинейная трапеция имеет более общий вид: $\{f_1(x)\leq y\leq f_2(x),\;a\leq x\leq b\}$ и $f_1(x)$, $f_2(x)$ --- непрерывны на $[a;b]$, тогда очевидно, что ее площадь:
  \begin{equation}
    S=\int_a^bf_2(x)dx-\int_a^bf_1(x)dx=\int_a^b(f_2(x)-f_1(x))dx
  \end{equation}
  
  \textbf{Пояснение 1:} Если граница трапеции пересекает ось абсцисс, нужно поднять ее вверх:
  \begin{equation}
    \overline{f_2}(x)=f_2(x)\;\;\overline{f_1}(x)=f_1(x)+d
  \end{equation}
  \begin{equation}
    S=\int_a^b(\overline{f_2}(x)+d)dx-\int_a^b(f_1(x)+d)dx=\int_a^b(f_2(x)-f_1(x))dx
  \end{equation}
  
  \textbf{Пояснение 2:} Если $f(x)<0$ на $[a;b]$, то
  \begin{equation}
    \int_a^bf(x)dx=-S
  \end{equation}
  
  \large\textbf{Площадь криволинейного сектора}\normalsize\\
  
  \textbf{Определение:} Пусть кривая $L$ задана в полярной системе координат, т.е. $L:\;r=r(\theta),\;\alpha\leq\theta\leq\beta$ и $r(\theta)\geq0$, $r(\theta)$ --- непрерывная функция. Плоскую фигуру, ограниченную кривой $L$ и двумя лучами, составляющими с полярной осью углы $\alpha$ и $\beta$ называют \textit{криволинейным сектором}.\\
  
  \textbf{Предложение:} криволинейный сектор --- квадрируемая фигура, площадь $\underline P$ которой вычисляется по формуле:
  \begin{equation}
    \underline P=\frac{1}{2}\int_\alpha^\beta r^2(\theta)d\theta
  \end{equation}
  
  \textbf{Доказательство:} пусть $\tau=\{\theta_k\}_{k=0}^n$ --- произвольное разбиение $[\alpha;\beta]$, $\alpha=\theta_0<\theta_1<...<\theta_n=\beta$.
  \begin{equation}
    \forall[\theta_{k-1};\theta_k]\;r_k=\min_{[\theta_{k-1};\theta_k]}; r(\theta)\; R_k=\max_{[\theta_{k-1};\theta_k]} r(\theta)
  \end{equation}
  
  Тогда:
  \begin{equation}
    \overline S_d=\frac{1}{2}\sum_{k=1}^nR_k^2\Delta\theta_k
  \end{equation}
  
  --- Площадь веерообразной фигуры, описанной вокруг криволинейного сектора, и:
  \begin{equation}
    \overline s_i=\frac{1}{2}\sum_{k=1}^nr_k^2\Delta\theta_k
  \end{equation}  
  
  --- Площадь веерообразной фигуры, вписанной в криволинейный сектор. Эти суммы как раз и есть $S=S(\tau)$ --- верхняя сумма Дарбу, и $s=s(\tau)$ --- нижняя сумма Дарбу для функции $\frac{1}{2}r^2(\theta)$ на отрезке $[\alpha;\beta]$, т.е. $\overline S_d=S$, $\overline s_i=s$.\\
  
  Так как $\frac{1}{2}r^2(\theta)$ непрерывна на $[\alpha;\beta]$, то и интегрируема на этом отрезке. Тогда по критерию интегрируемости в допредельной форме:
  \begin{equation}
    \forall\varepsilon>0\;\exists\tau=\{\theta_k\}_{k=0}^n \text{ --- разбиение отрезка,}:S(\tau)-s(\tau)<\frac{\varepsilon}{2}
  \end{equation}
  
  Значит:
  \begin{equation}
    \overline S_d-\overline s_i=S(\tau)-s(\tau)<\frac{\varepsilon}{2}
  \end{equation}
  
  Поскольку существует площадь кругового сектора (из школьного курса математики), то по определению площади для этого $\varepsilon>0$ существует многоугольник $Q_i$, вписанный в нижнюю веерообразную фигуру. Его площадь $s_i:$
  \begin{equation}
    \overline s_i-s_i<\frac{\varepsilon}{4}
  \end{equation}
  
  Аналогично $\exists Q_d$ --- многоугольник, описанный вокруг верхней веерообразной фигуры площади $S_d:$
  \begin{equation}
    S_d-\overline S_d<\frac{\varepsilon}{4}
  \end{equation}
  
  \begin{equation}
    S_d-s_i=S_d-\overline S_d+\overline S_d-\overline s_i+\overline s_i-s_i<\varepsilon
  \end{equation}
  
  По критерию квадрируемости следует, что криволинейный сектор --- квадрируемая фигура.
  
  \begin{equation}
    \overline s_i\rightarrow \frac{1}{2}\int_\alpha^\beta r^2(\theta)d\theta\;\;\;\overline S_d\rightarrow \frac{1}{2}\int_\alpha^\beta r^2(\theta)d\theta
  \end{equation}
  
  Поскольку $s_i$ и $S_d$ отличаются как угодно мало от $\overline s_i$ и $\overline S_d$ соответственно, то:
  \begin{equation}
    P=\frac{1}{2}\int_\alpha^\beta r^2(\theta)d\theta
  \end{equation}
  
  \textbf{Предложение доказано.}\\
  
  \textbf{Пример:}\\
  
  Вычислить площадь фигуры F, ограниченной графиками функций $y=x^\alpha$ и при $\alpha>1\;\;x=y^\alpha$. Решение:
  
  \begin{equation}
    S=1-2\int_0^1 x^\alpha dx=\frac{\alpha-1}{\alpha+1}
  \end{equation}
  
  \huge\textbf{Объем тела}\normalsize\\
  
  \textbf{Определение:} \textit{телом} назовем часть пространства, ограниченную замкнутой непересекающейся поверхностью.\\
  
  \textbf{Определение:} \textit{многогранник} --- это часть пространства, ограниченная частями плоскостей.\\
  
  Пусть есть тело $E$. Рассмотрим всевозможные многогранники, вписанные в тело $E$ и всевозможные многогранники, описанные вокруг этого тела. Вычисление объема многогранника сводится к вычислению объемов тетраэдров, поэтому понятие объема многогранники считаем известным из школьного курса математики.\\
  
  Пусть $\{v_i\}$ --- числовое множество объемов, вписанных в тело $E$ многогранников, а $\{V_d\}$ --- числовое множество объемов описанных вокруг тела $E$ многогранников. \\
  
  Множество $\{v_i\}$ ограничено сверху любым объемом описанного многогранника, а множество $\{V_d\}$ ограничено снизу любым объемом вписанного многогранника, или даже нулем; значит $\exists\underline v=\sup\{v_i\}$ и $\exists\overline V=\inf\{V_d\}$. Числа $\underline v$ и $\overline V$ называются соответственно \textit{нижним} и \textit{верхним} объемами тела $E$.\\
  
  Очевидно, $\underline v\leq \overline V$.\\
  
  \textbf{Определение:} тело $E$ называется \textit{кубируемым}, если верхний объем $\overline V$ этого тела совпадает с нижним объемом $\underline v$ и при этом $V=\overline V=\underline v$ называется \textit{объемом тела $E$}.\\
  
  \large\textbf{Критерий кубируемости фигуры}\normalsize\\
  
  \textbf{Теорема:} для того, чтобы тело $E$ было кубируемым, необходимо и достаточно, чтобы для любого $\varepsilon>0$ можно было указать такой описанный около тела $E$ многогранник и такой вписанный в это тело многогранник, разность объемов $V_d-v_i$ которых была бы меньше $\varepsilon$:
  \begin{equation}
    E\text{--- кубируема} \;\Leftrightarrow \;\forall\varepsilon>0\;\exists\text{многогр-ки}:V_d-v_i<\varepsilon
  \end{equation}
  
  \textbf{Доказательство:} аналогично плоскому случаю.\\
  
  \huge\textbf{Цилиндр}\normalsize\\
  
  \textbf{Определение:} \textit{цилиндр} --- это тело, ограниченное цилиндрической поверхностью и образующими, параллельными некоторой оси и плоскостями, перпендикулярными некоторой оси; эти плоскости в пересечении с цилиндрическими поверхностями образуют плоские фигуры, называемые основаниями цилиндра, а расстояние $h$ между основаниями цилиндра называют \textit{высотой цилиндра}.\\
  
  Показанная выше техника позволяет просто доказать следующие утверждения:\\
  
  \textbf{Утверждение 1:} если основанием цилиндра $E$ является квадрируемая фигура $Q$, то цилиндр является кубируемым телом и его объем равен $V=Ph$, где $P$ --- площадь основания, а $h$ --- высота цилиндра.\\
  
  \textbf{Определение:} ступенчатым телом называется объединение конечного числа цилиндров, расположенных так, что верхнее основание каждого предыдущего из этих цилиндров находится в одной плоскости с нижним основанием последующего цилиндра.\\
  
  \textbf{Утверждение 2:} если для любого $\varepsilon>0$ можно указать такое описанное вокруг $E$ ступенчатое тело и такое вписанное в $E$ ступенчатое тело, разность $V_d-v-i$ объемов которых меньше $\varepsilon$, то тело кубируемо.\\
  
  \textbf{Утверждение 3:} пусть функция $y=f(x)$ непрерывна на $[a;b]$, тогда тело $E$, образованное вращением вокруг оси $Ox$ криволинейной трапеции, ограниченной графиком функции $y=f(x)$, ординатами в точках $a$ и $b$, кубируема, и его объем вычисляется по формуле:
  \begin{equation}
    V=\pi\int_a^b f^2(x)dx
  \end{equation}
  
  \textbf{Доказательство:} пусть $\tau=\{x_k\}_{k=0}^n,\;a=x_0<x_1<...<x_n=b$ --- произвольное разбиение отрезка $[a;b]$. Пусть:
  \begin{equation}
    m_i=\inf_{[x_{i-1};x_i]} f(x);\;\;\;M_i=\sup_{[x_{i-1};x_i]} f(x)
  \end{equation}
  
  Построим на каждом частичном отрезке прямоугольники. Получили вписанную в криволинейную трапецию и описанную около нее ступенчатую фигуру, при вращении которой вокруг оси $Ox$ получаем вписанное в $E$ и описанное вокруг него тела, объемы которых равны:
  \begin{equation}
    v_i=\pi\sum_{i=1}^n m_i^2\Delta x_i;\;\;V_d=\pi\sum_{i=1}^n M_i^2\Delta x_i
  \end{equation}
  
  Очевидно, что $v_i=s(\tau)$ --- нижняя, а $V_d=S(\tau)$ --- верхняя суммы Дарбу для функции $\pi f^2(x)$ на $[a;b]$ и в силу непрерывности $f(x)$ эта функция и интегрируема на отрезке $[a;b]$. Тогда по критерию интегрируемости:
  \begin{equation}
    \exists\tau=\{x_i\}_{i=0}^n \text{ --- разбиение }[a;b]:\;\; V_d-v_i=S(\tau)-s(\tau)<\varepsilon
  \end{equation}
  А это по утверждению 2 влечет кубируемость тела $E$ и очевидно следующее:
  
  \begin{equation}
    V=\pi\int_a^b f^2(x)dx
  \end{equation}
  
  \textbf{Утверждение доказано.}\\
  
  \textbf{Пример:}\\
  
  $y=\sin x$, $[0;\pi]$, вычислить объем тела вращения графика вокруг оси $Ox$. Решение:\\
  
  \begin{equation}
    V=\pi\int_0^p\sin^x dx=\pi\int_0^pi\frac{1-\cos 2x}{2}dx=\frac{\pi^2}{2}
  \end{equation}
  
  \large\textbf{Площадь поверхности вращения}\normalsize\\
  
  Рассмотрим поверхность $\pi$, образованную вращением вокруг $Ox$ графика функции $y=f(x),\;x\in[a;b]$. Пусть $\tau=\{x_i\}_{i=0}^n=\{a=x_0<x_1<...<x_n=b\}$ --- разбиение $[a;b]$. Рассмотрим также $A:(x_i,y_i)$, где $y_i$ --- значение функции $f(x_i)\;y_i=f(x_i)$.\\
  
  Соединим точки $A_i$ отрезками. Получим ломаную, соответствующую разбиению $\tau$. Вращая график $f(x)$ и ломаную вокруг $Ox$, получим поверхность $\pi$ --- поверхность вращения графика функции $f(x)$ и поверхность $\Pi$ --- поверхность, полученная вращением ломаной $A_0,\;A_1,\;A_n$ вокруг $Ox$. Обозначим через $P(\tau)$ поверхность $\Pi(\tau)$. Пусть $l$ --- длина звена $[A_{i-1};A]$ ломаной. Тогда по формулам элементарной математики (усеченного конуса):\\
  \begin{equation}
    P(\tau)=2\pi\sum_{i=1}^n\frac{y_{i-1}+y_i}{2}l_i=\pi\sum_{i=1}^n(y_{i-1}+y_i)l_i
  \end{equation}
  
  \textbf{Определение:} число $P$ является пределом площадей $P(\tau)$, если:
  \begin{equation}
    \forall\varepsilon>0\;\exists\delta=\delta(\varepsilon)>0:\;\forall\tau=\{x_i\}_{i=0}^n:\;x(\tau)<\delta\Rightarrow |P(\tau)-P|<\varepsilon
  \end{equation}
  
  
\end{document}